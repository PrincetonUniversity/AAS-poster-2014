\documentclass[blockbodyinnersep=1cm,titleinnersep=1cm,20pt]{tikzposter}
\tikzposterlatexaffectionproofoff
\usecolortheme{armin}
\usepackage{siunitx}
\usepackage{natbib}
\usepackage{lmodern}
\usepackage{graphicx}
\usepackage{caption}
\usepackage{wrapfig}
\usepackage{aas_macros,hyperref,amsmath}
\graphicspath{ {./figures/} }

\begingroup
\renewcommand{\section}[2]{}
\newcommand{\flash}{\textsc{flash}}

\defcitealias{2009ApJ...700.1148D}{Paper~I}

 % Title, Author, Institute
\title{Investigating mass transfer in symbiotic systems with hydrodynamic
  simulations}
\author{
  M.~de~Val-Borro\textsuperscript{1},
  M.~Karovska\textsuperscript{2},
  and D.~D.~Sasselov\textsuperscript{2}
}
\institute{
  \textsuperscript{1}Princeton University,
  \textsuperscript{2}Harvard-Smithsonian Center for Astrophysics
}

 % Begin document
\begin{document}

 % Title block
\titleblock[left fig=/home/mdevalbo/Princeton/identity/cfa-logo_146.jpg,
	    left fig height=7cm,
	    right fig=/home/mdevalbo/Princeton/identity/Princeton_shield.pdf,
	    right fig height=7cm,
	    embedded=false, separated=false]

\block[c,width=0.7\textwidth]{Abstract}{
We investigate gravitationally focused wind accretion in binary systems
consisting of an evolved star with a gaseous envelope and a compact
accreting companion.  We study the mass accretion and formation of an
accretion disk around the secondary caused by the strong wind from the
primary late-type component using global 2D and 3D hydrodynamic
numerical simulations.  In particular, the dependence on the mass
accretion rate on the  mass loss rate, wind temperature and orbital
parameters of the system is considered.  For a typical slow and massive
wind from an evolved star the mass transfer through a focused wind
results in rapid infall onto the secondary. A stream flow is created
between the stars with accretion rates of a 2-10\% percent of the mass
loss from the primary.  This mechanism could be an important method for
explaining periodic modulations in the accretion rates for a broad range
of interacting binary systems and fueling of a large population of X-ray
binary systems.  We test the plausibility of these accretion flows
indicated by the simulations by comparing with observations of the
symbiotic variable system CH Cyg.

}

 \begin{columns}
 % Set first column
\column{0.5}

 % First column - first block
\block{Symbiotic binaries}{
Symbiotic binaries are important astrophysical laboratories for studies
of wind accretion because of the wide separation of the components, and
the ability to study the individual components and the accretion
processes at many wavelengths ranging from X-ray to radio
\citep{1997ApJ...482L.175K,2005ApJ...623L.137K,2006ApJ...637L..49M}.  A
typical symbiotic consists of a mass-losing AGB or a red giant star and
a hot accreting companion, often a white dwarf (WD).  The components in
these systems are assumed to be detached (at least during most of the
orbital motion) and the compact companion accretes mass from the massive
wind of the cool evolved star.

}

\block{Numerical model}{
The numerical model is based on previous two-dimensional models
described in \citet{2009ApJ...700.1148D}, hereafter referred to as
\citetalias{2009ApJ...700.1148D}.
We solve the basic equations of hydrodynamics describing the evolution
of the density and velocity field in three-dimensions:
\begin{align}
  \frac{\partial\rho}{\partial t} + \nabla \cdot (\rho \mathbf{v}) & = 0,\\
  \frac{\partial\mathbf{v}}{\partial t} + (\mathbf{v} \cdot \nabla )
  \mathbf{v} & = - \frac{1}{\rho}\nabla p - \nabla \Phi,
  \label{eq:NS}
\end{align}
where $\rho$ is the density in the orbital plane, $\mathbf{v}$ the
velocity of the fluid, $p$ the pressure and $\Phi$ the gravitational
potential.

Our code is based on the \flash{} code \citep{2000ApJS..131..273F},
which is a fully parallel block-structured Adaptive Mesh Refinement
(AMR) implementation of the Piecewise Parabolic Method
\citep[PPM;][]{1984JCoPh..54..115W,1984JCoPh..54..174C} in its original
Eulerian form\footnote{The source code is available at
\url{http://www.flash.uchicago.edu/}}.
The code has been extensively tested in various compressible flow
problems with astrophysical applications \citep[see
e.g.][]{2002ApJS..143..201C,2006MNRAS.370..529D}.

}

\block{Results}{
}

 % Set second column
\column{0.5}

% First column - third block
\block{Data analysis}{

}

\block[width=\colwidth]{References}{
\bibliographystyle{aa}
\bibliography{ads}
}

\end{columns}

\end{document}
