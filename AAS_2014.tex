% vim: set spell spelllang=en_us:
\documentclass[blockbodyinnersep=1cm,titleinnersep=1cm]{tikzposter}
\tikzposterlatexaffectionproofoff
\usecolortheme{armin}
\usepackage{siunitx}
\usepackage{natbib}
\usepackage{lmodern}
\usepackage{graphicx}
\usepackage{caption}
\usepackage{wrapfig}
\usepackage{aas_macros,hyperref,amsmath}
\graphicspath{
  {/home/mdevalbo/FLASH/FLASH2.5/runs/mira-isot-della4/}
  {./figures/} }

\begingroup
\renewcommand{\section}[2]{}
\newcommand{\flash}{\textsc{flash}}
\DeclareSIUnit\year{yr}

\defcitealias{2009ApJ...700.1148D}{Paper~I}

 % Title, Author, Institute
\title{Investigating mass transfer in wind-accreting symbiotic systems with
  AMR hydrodynamic simulations}
\author{
  M.~de~Val-Borro\textsuperscript{1},
  M.~Karovska\textsuperscript{2},
  and D.~D.~Sasselov\textsuperscript{2}
}
\institute{
  \textsuperscript{1}Princeton University,
  \textsuperscript{2}Harvard-Smithsonian Center for Astrophysics
}

 % Begin document
\begin{document}

 % Title block
\titleblock[left fig=/home/mdevalbo/Princeton/identity/cfa-logo_146.jpg,
	    left fig height=7cm,
	    right fig=/home/mdevalbo/Princeton/identity/Princeton_shield.pdf,
	    right fig height=7cm,
	    embedded=false, separated=false]

\block[c,width=0.7\textwidth]{Abstract}{
We investigate gravitationally focused wind accretion in binary systems
consisting of an evolved star with a gaseous envelope and a compact
accreting companion.  We study the mass accretion and formation of an
accretion disk around the secondary caused by the strong wind from the
primary late-type component using global 2D and 3D hydrodynamic
numerical simulations.  In particular, the dependence on the mass
accretion rate on the  mass loss rate, wind temperature and orbital
parameters of the system is considered.  For a typical slow and massive
wind from an evolved star the mass transfer through a focused wind
results in rapid infall onto the secondary. A stream flow is created
between the stars with accretion rates of a 2-10\% percent of the mass
loss from the primary.  This mechanism could be an important method for
explaining periodic modulations in the accretion rates for a broad range
of interacting binary systems and fueling of a large population of X-ray
binary systems.  We test the plausibility of these accretion flows
indicated by the simulations by comparing with observations of the
symbiotic variable system CH Cyg.
}

\begin{columns}
 % Set first column
\column{0.5}

 % First column - first block
\block{Symbiotic binaries}{
Symbiotic binaries are unique astrophysical laboratories for studies
of wind accretion because of the wide separation of the components, and
the ability to study the individual components and the accretion
processes from multiwavelength studies from X-ray to radio
\citep[e.g.,][]{2010ApJ...710L.132K}.  A typical symbiotic consists of a
mass-losing AGB or a red giant star and a hot accreting companion, often a
white dwarf (WD).  The components in these systems are assumed to be detached
(at least during most of the orbital motion) and the compact companion accretes
mass from the massive wind of the cool evolved star.
}

\block{Numerical model}{
The numerical model is based on previous two-dimensional models
described in \citet{2009ApJ...700.1148D}, hereafter referred to as
\citetalias{2009ApJ...700.1148D}.
We solve the basic equations of hydrodynamics describing the evolution
of the density and velocity field in three-dimensions:
\begin{align}
  \frac{\partial\rho}{\partial t} + \nabla \cdot (\rho \mathbf{v}) & = 0,\\
  \frac{\partial\mathbf{v}}{\partial t} + (\mathbf{v} \cdot \nabla )
  \mathbf{v} & = - \frac{1}{\rho}\nabla p - \nabla \Phi,
  \label{eq:NS}
\end{align}
where $\rho$ is the density in the orbital plane, $\mathbf{v}$ the
velocity of the fluid, $p$ the pressure and $\Phi$ the gravitational
potential.

Our code is based on the \flash{} code \citep{2000ApJS..131..273F}, which is a
fully parallel block-structured Adaptive Mesh Refinement (AMR) implementation
of the Piecewise Parabolic Method (PPM) in its original Eulerian
form\footnote{The source code is available at
\url{http://www.flash.uchicago.edu/}}.  The code has been extensively tested in
various compressible flow problems with astrophysical applications \citep[see
e.g.][]{2006MNRAS.370..529D}.
}

\block{Results}{
We present numerical simulations of gravitationally focused wind accretion in
the CH Cyg binary system.  The wind parameters of our model are those of a slow
spherically symmetric wind from an evolved star at the dust acceleration
surface.  We have considered simulations with a locally isothermal equation of
state.  The results of this work and a comparison of the models with
observations of the symbiotic binary CH Cyg are summarized below:
\begin{itemize}
  \item The dynamics of wind accretion in this system show a complex dynamical
    caused by the presence of a companion.
  \item A bow shock and a tail structure is formed at the position of the
secondary due to the supersonic orbital motion.
  \item Mass transfer through a focused wind leads to a stream flow onto the
WD with relative accretion rates of a 2-10\% percent of the mass loss
from the primary.
  \item This mechanism could explain the jets detected in HST and Chandra
images of CH Cyg \citep{2010ApJ...710L.132K}.
\end{itemize}
}

\block{Acknowledgments}{
We acknowledge partial support from grants NSF AST-1108686 and NASA
NNX12AH91H.  The numerical code used in this work is based on the \flash{} code
that is in part developed by the DOE NNSA-ASC OASCR Flash Center at the
University of Chicago.
}

 % Set second column
\column{0.5}

% First column - third block
\block{Wind accretion in CH~Cyg}{
\begin{wrapfigure}{r}{0.6\textwidth}
  \centering
  \includegraphics[width=.6\colwidth]{chcyg_2D_hdf5_plt_cnt_0101.png}
  \caption*{Orbital plane density contours in logarithmic scale for a
    simulation of CH~Cyg after an orbital period.  The system has a separation
    of \SI{10}{AU} and a mass ratio $q=3.3$.}
\end{wrapfigure}
We integrate the equations of hydrodynamics until the density distribution of
the gravitationally focused wind reaches a quasi-stationary state.  The
adopted orbital parameters correspond to the symbiotic binary CH~Cyg:
$M_1=2M_\odot$,
$M_2=0.6M_\odot$ and
$P = \SI{15.6}{\year}$
\citep[][and references therein]{2009MNRAS.397..325P}.
The stellar
wind produced by the donor star is deflected towards the secondary within the
orbital plane, and the relaxed density distributions present Keplerian
accretion disks of various sizes.  A tidal stream is observed in our
simulations for a wide range of binary separations Using an isothermal equation
of state.  The formation of a stream flow is dependent on the wind velocity and
temperature at the dust formation radius which determines the velocity at the
secondary's position.

\begin{wrapfigure}{l}{0.6\textwidth}
  \centering
  \includegraphics[width=.6\colwidth]{proj_1orb_obs2008Ccrop.jpeg}
  \caption*{
  HST image of CH Cyg acquired in 2008 \citep{2010ApJ...710L.132K}
  and the projected simulation taking into account the inclination
  ($\sim$ \SI{80}{\degree}) with respect to the line of sight.
}
\end{wrapfigure}
Our simulations were run on a grid with uniform radial spacing in our base grid
and 5 additional levels of refinement.  The computational grid was centered on
the giant star in the rotating reference frame of the binary system.  The
numerical method changes the accuracy of the solution by modifying the spacing
of the grid points in certain regions while the system is evolved.  The AMR
algorithm places high resolution grid patches only at the regions where
time-dependent shocks are formed.  An HST image of CH Cyg with a field of view
of $5 \times 3.5$ \si{\arcsecond} is shown in the left panel of the figure on
the left. The cyan box in the center of the observed CH Cyg image has
approximately the same size of the inner region showed in the simulated image
in the right panel with a field of view of approximately $45 \times 30$ AU.
There is a strikingly similar morphology of the accretion outflow on the
companion and beyond with that of the observed outflow structure at much larger
scales.}

\block[width=\colwidth]{References}{
\bibliographystyle{aa}
\bibliography{ads}
}

\end{columns}

\end{document}
