% vim: set spell spelllang=en_us:
\documentclass[blockbodyinnersep=1cm,titleinnersep=1cm]{tikzposter}
\tikzposterlatexaffectionproofoff
\usecolortheme{nicolas}
\usepackage{siunitx}
\usepackage{natbib}
\usepackage{lmodern}
\usepackage{graphicx}
\usepackage{caption}
\usepackage{wrapfig}
\usepackage{aas_macros,hyperref,amsmath}
\usepackage{booktabs}
\graphicspath{
  {/home/mdevalbo/FLASH/FLASH2.5/runs/mira-isot-della4/}
  {./figures/} }

\begingroup
\renewcommand{\section}[2]{}
\newcommand{\flash}{\textsc{flash}}
\DeclareSIUnit\year{yr}

\defcitealias{2009ApJ...700.1148D}{Paper~I}

 % Title, Author, Institute
\title{\parbox{\linewidth}{\centering
  Investigating mass transfer in symbiotic systems with
  hydrodynamic simulations\vspace{.5cm}}
}
\author{
  M.~de~Val-Borro\textsuperscript{1},
  M.~Karovska\textsuperscript{2},
  and D.~D.~Sasselov\textsuperscript{2}
}
\institute{
  \textsuperscript{1}Princeton University,
  \textsuperscript{2}Harvard-Smithsonian Center for Astrophysics
}

 % Begin document
\begin{document}

 % Title block
\titleblock[left fig=/home/mdevalbo/Princeton/identity/cfa-logo_146_alpha.png,
	    left fig height=7cm,
	    right fig=/home/mdevalbo/Princeton/identity/Princeton_shield.pdf,
	    right fig height=7cm,
	    embedded=false, separated=false]

\block[c,width=0.7\textwidth]{Abstract}{
We investigate gravitationally focused wind accretion in binary systems
consisting of an evolved star with a gaseous envelope and a compact
accreting companion.  We study the mass accretion and formation of an
accretion disk around the secondary caused by the strong wind from the
primary late-type component using global 2D and 3D hydrodynamic
numerical simulations.  In particular, the dependence of the mass
accretion rate on the mass loss rate, wind temperature and orbital
parameters of the system is considered.  For a typical slow and massive
wind from an evolved star the mass transfer through a focused wind
results in rapid infall onto the secondary. A stream flow is created
between the stars with accretion rates of a 2--10\% percent of the mass
loss from the primary.  This mechanism could be an important method for
explaining periodic modulations in the accretion rates for a broad range
of interacting binary systems and fueling of a large population of X-ray
binary systems.  We test the plausibility of these accretion flows
indicated by the simulations by comparing with observations of the
symbiotic variable system CH Cyg.
}

\begin{columns}
 % Set first column
\column{0.5}

 % First column - first block
\block{Symbiotic binaries}{
Symbiotic binaries are unique astrophysical laboratories for studies
of wind accretion because of the wide separation of the components, and
the ability to study the individual components and the accretion
processes from multiwavelength studies from X-ray to radio
\citep[e.g.,][]{2010ApJ...710L.132K}.  A typical symbiotic consists of a
mass-losing AGB or a red giant star and a hot accreting companion, often a
white dwarf (WD).  The components in these systems are assumed to be detached
(at least during most of the orbital motion) and the compact companion accretes
mass from the massive wind of the cool evolved star.
}

\block{Numerical model}{
The numerical model is based on previous two-dimensional models
described in \citet{2009ApJ...700.1148D}, hereafter referred to as
\citetalias{2009ApJ...700.1148D}.
We solve the basic equations of hydrodynamics describing the evolution
of the density and velocity field in three-dimensions:
\begin{align}
  \frac{\partial\rho}{\partial t} + \nabla \cdot (\rho \mathbf{v}) & = 0,\\
  \frac{\partial\mathbf{v}}{\partial t} + (\mathbf{v} \cdot \nabla )
  \mathbf{v} & = - \frac{1}{\rho}\nabla p - \nabla \Phi,
  \label{eq:NS}
\end{align}
where $\rho$ is the density in the orbital plane, $\mathbf{v}$ the
velocity of the fluid, $p$ the pressure and $\Phi$ the gravitational
potential.

Our code is based on the \flash{} code \citep{2000ApJS..131..273F}, which is a
fully parallel block-structured Adaptive Mesh Refinement (AMR) implementation
of the Piecewise Parabolic Method (PPM) in its original Eulerian
form\footnote{The source code is available at
\url{http://www.flash.uchicago.edu/}}.  The code has been extensively tested in
various compressible flow problems with astrophysical applications.  A
realistic 3D hydrodynamical simulations of the interaction of the wind from a
giant star with a companion using the spherical coordinates version of the code
is in progress. The computational domain spans about one radian around the
orbital plane of the system centered on the primary to cover the whole region
of interest close to the accretor. The spatial resolution per separation in the
3D simulations is about a factor of two smaller than in the 2D simulations.
}

\block{Results}{
We present numerical simulations of gravitationally focused wind accretion in
the CH Cyg binary system.  The wind parameters of our model are those of a slow
spherically symmetric wind from an evolved star at the dust acceleration
surface.  A tidal stream is observed in our simulations for a wide range of
binary separations using an isothermal equation of state.  We have considered
simulations with a locally isothermal equation of state.  The results of this
work and a comparison of the models with observations of the symbiotic binary
CH Cyg are summarized below:
\begin{itemize}
  \item The focused wind accretion in this system shows a complex dynamics,
    resulting in a powerful flow beyond the accreting companion.
  \item A bow shock and a spiral-like tail structure forms near the position
    of the secondary and on the farther side from the primary  due to the
    supersonic orbital motion.
  \item Mass transfer through a focused wind leads to a stream flow onto the WD
    with variable accretion rates of 2--10\% percent of the mass loss
    from the primary.
  \item This mechanism could explain the morphology of the outflows detected in
    the HST and Chandra images of CH Cyg at much larger scales
    \citep[e.g.,][]{2010ApJ...710L.132K}.
\end{itemize}
}

 % Set second column
\column{0.5}

% First column - third block
\block{Wind accretion in CH~Cygni}{
\begin{wraptable}{l}{0.4\textwidth}
\centering
\caption*{CH Cyg system}
\begin{tabular}{ccc}
  \toprule
  Parameter & Value \\
  \midrule
  $M_\mathrm{RG}$ & $2M_\odot$\\
  $M_\mathrm{WD}$ & $0.6M_\odot$\\
  $P$ & \SI{15.6}{\year}\\
  $a$ & \SI{8.5}{AU}\\
  $\dot M$ & \num{2e-6} $M_\odot$/yr\\
  $v$ & 20--40 \si{\km\per\s}\\
  \bottomrule
\end{tabular}
\end{wraptable}
We integrate the equations of hydrodynamics until the density distribution of
the gravitationally focused wind reaches a quasi-stationary state.  The
adopted orbital and wind parameters correspond to the symbiotic binary CH~Cyg
\citep[long-period orbit in][shown in the table on the
left]{2009ApJ...692.1360H}.
The stellar wind produced by the donor star is deflected towards the secondary
within the orbital plane, and the relaxed density distributions present
a Keplerian accretion disk.

\begin{wrapfigure}{r}{0.6\textwidth}
  \centering
  \includegraphics[width=.6\colwidth]{chcyg_2D_hdf5_plt_cnt_0101.png}
  \caption*{Orbital plane density contours in logarithmic scale for a
    simulation of CH~Cyg after an orbital period.  The system has a separation
    of \SI{8.5}{AU} and a mass ratio $q=3.3$.}
\end{wrapfigure}
The formation of a stream flow is dependent on the wind velocity and
temperature at the dust formation radius which determines the velocity at the
secondary's position.  Our simulations were run on a grid with uniform radial
spacing in our base grid and 5 additional levels of refinement.  The
computational grid was centered on the giant star in the rotating reference
frame of the binary system.  The numerical method changes the accuracy of the
solution by modifying the spacing of the grid points in certain regions
as a function of time.  The AMR algorithm places high resolution grid patches
only at the regions where time-dependent shocks are formed.

\begin{wrapfigure}{l}{0.6\textwidth}
  \centering
  \includegraphics[width=.6\colwidth]{proj_1orb_obs2008Ccrop.jpeg}
  \caption*{
  HST image of CH Cyg acquired in 2008 \citep{2010ApJ...710L.132K}
  with FOV of 1250 AU $\times$ 875 AU
  and the projected simulation taking into account the inclination
  ($\sim$ \SI{84}{\degree}) with respect to the line of sight
  with a FOV of approximately 45 AU $\times$ 30 AU.
}
\end{wrapfigure}
An HST image of CH Cyg is shown in the left panel of the figure on
the left. The cyan box in the center of the observed CH Cyg image has
approximately the same size of the inner region showed in the simulated image
in the right panel.
There is a strikingly similar morphology of the accretion outflow on the
companion and beyond with that of the observed outflow structure at much larger
scales.}

\block{Acknowledgments}{
Based on observations made with the NASA/Chandra and NASA/ESA HST (obtained
at the STScI, which is operated by the AURA, Inc.).  The numerical code used in
this work is based on the \flash{} code that is in part developed by the DOE
NNSA-ASC OASCR Flash Center at the University of Chicago.
}

\block[width=\colwidth]{References}{
\bibliographystyle{aa}
\bibliography{ads}
}

\end{columns}

\end{document}
